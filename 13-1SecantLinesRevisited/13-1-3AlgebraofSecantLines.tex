\documentclass{ximera}

\graphicspath{
  {./}
  {1-1QuantitativeReasoning/}
  {1-2RelationsAndGraphs/}
  {1-3ChangingInTandem/}
  {2-1LinearEquations/}
  {2-2LinearModeling/}
  {2-3ExponentialModeling/}
  {3-1WhatIsAFunction/}
  {3-2FunctionProperties/}
  {3-3AverageRatesOfChange/}
  {4-1BuildingNewFunctions/}
  {4-2Polynomials/}
  {5-1RationalFunctions/}
   {5-2ExponentialFunctions/}
  {6-1Domain/}
  {6-2Range/}
  {6-3CompositionOfFunctions/}
  {6-4FunctionTransformations/}
  {7-1ZerosOfFunctions/}
  {7-XZerosOfPolynomials/}
  {7-2ZerosOfFamousFunctions/}
  {8-1SystemsOfEquations/}
  {6-5FunctionTransformationsProject/}
  {1-1QuantitativeReasoning/exercises/}
  {1-2RelationsAndGraphs/exercises/}
  {../1-3ChangingInTandem/exercises/}
  {../2-1LinearEquations/exercises/}
  {../2-2LinearModeling/exercises/}
  {../2-3ExponentialModeling/exercises/}
  {../3-1WhatIsAFunction/exercises/}
  {../3-2FunctionProperties/exercises/}
  {../3-3AverageRatesOfChange/exercises/}
  {../5-2ExponentialFunctions/exercises/}
  {../4-1BuildingNewFunctions/exercises/}
  {../4-2Polynomials/exercises/}
  {../5-1RationalFunctions/exercises/}
  {../6-1Domain/exercises/}
  {../6-2Range/exercises/}
  {../6-3CompositionOfFunctions/exercises/}
  {../7-1ZerosOfFunctions/exercises/}
  {../7-XZerosOfPolynomials/exercises/}
  {../7-2ZerosOfFamousFunctions/exercises/}
  {../6-4FunctionTransformations/exercises/}
  {../8-1SystemsOfEquations/exercises/}
  {../6-3FunctionTransformationsProject/exercises/}
}

\DeclareGraphicsExtensions{.pdf,.png,.jpg,.eps}

\newcommand{\mooculus}{\textsf{\textbf{MOOC}\textnormal{\textsf{ULUS}}}}

\usepackage[makeroom]{cancel} %% for strike outs

\ifxake
\else
\usepackage[most]{tcolorbox}
\fi


%\typeout{************************************************}
%\typeout{New Environments}
%\typeout{************************************************}

%% to fix for web can be removed when deployed offically with ximera2
\let\image\relax\let\endimage\relax
\NewEnviron{image}{% 
  \begin{center}\BODY\end{center}% center
}



\NewEnviron{folder}{
      \addcontentsline{toc}{section}{\textbf{\BODY}}
}

\ifxake
\let\summary\relax
\let\endsummary\relax
\newtheorem*{summary}{Summary}
\newtheorem*{callout}{Callout}
\newtheorem*{overview}{Overview}
\newtheorem*{objectives}{Objectives}
\newtheorem*{motivatingQuestions}{Motivating Questions}
\newtheorem*{MM}{Metacognitive Moment}
      
%% NEEDED FOR XIMERA 2
%\ximerizedEnvironment{summary}
%\ximerizedEnvironment{callout}
%\ximerizedEnvironment{overview} 
%\ximerizedEnvironment{objectives}
%\ximerizedEnvironment{motivatingQuestions}
%\ximerizedEnvironment{MM}
\else
%% CALLOUT
\NewEnviron{callout}{
  \begin{tcolorbox}[colback=blue!5, breakable,pad at break*=1mm]
      \BODY
  \end{tcolorbox}
}
%% MOTIVATING QUESTIONS
\NewEnviron{motivatingQuestions}{
  \begin{tcolorbox}[ breakable,pad at break*=1mm]
    \textbf{\Large Motivating Questions}\hfill
    %\begin{itemize}[label=\textbullet]
      \BODY
    %\end{itemize}
  \end{tcolorbox}
}
%% OBJECTIVES
\NewEnviron{objectives}{  
    \vspace{.5in}
      %\begin{tcolorbox}[colback=orange!5, breakable,pad at break*=1mm]
    \textbf{\Large Learning Objectives}
    \begin{itemize}[label=\textbullet]
      \BODY
    \end{itemize}
    %\end{tcolorbox}
}
%% DEFINITION
\let\definition\relax
\let\enddefinition\relax
\NewEnviron{definition}{
  \begin{tcolorbox}[ breakable,pad at break*=1mm]
    \noindent\textbf{Definition}~
      \BODY
  \end{tcolorbox}
}
%% OVERVIEW
\let\overview\relax
\let\overview\relax
\NewEnviron{overview}{
  \begin{tcolorbox}[ breakable,pad at break*=1mm]
    \textbf{\Large Overview}
    %\begin{itemize}[label=\textbullet] %% breaks Xake
      \BODY
    %\end{itemize}
  \end{tcolorbox}
}
%% SUMMARY
\let\summary\relax
\let\endsummary\relax
\NewEnviron{summary}{
  \begin{tcolorbox}[ breakable,pad at break*=1mm]
    \textbf{\Large Summary}
    %\begin{itemize}[label=\textbullet] %% breaks Xake
      \BODY
    %\end{itemize}
  \end{tcolorbox}
}
%% REMARK
\let\remark\relax
\let\endremark\relax
\NewEnviron{remark}{
  \begin{tcolorbox}[colback=green!5, breakable,pad at break*=1mm]
    \noindent\textbf{Remark}~
      \BODY
  \end{tcolorbox}
}
%% EXPLANATION
\let\explanation\relax
\let\endexplanation\relax
\NewEnviron{explanation}{
    \normalfont
    \noindent\textbf{Explanation}~
      \BODY
}
%% EXPLORATION
\let\exploration\relax
\let\endexploration\relax
\NewEnviron{exploration}{
  \begin{tcolorbox}[colback=yellow!10, breakable,pad at break*=1mm]
    \noindent\textbf{Exploration}~
      \BODY
  \end{tcolorbox}
}
%% METACOGNITIVE MOMENTS
\let\MM\relax
\let\endMM\relax
\NewEnviron{MM}{
  \begin{tcolorbox}[colback=pink!15, breakable,pad at break*=1mm]
    \noindent\textbf{Metacognitive Moment}~
      \BODY
  \end{tcolorbox}
}


\fi





%Notes on what envirnoment to use:  Example with Explanation in text; if they are supposed to answer- Problem; no answer - Exploration


%\typeout{************************************************}
%% Header and footers
%\typeout{************************************************}

\newcommand{\licenseAcknowledgement}{Licensed under Creative Commons 4.0}
\newcommand{\licenseAPC}{\renewcommand{\licenseAcknowledgement}{\textbf{Acknowledgements:} Active Prelude to Calculus (https://activecalculus.org/prelude) }}
\newcommand{\licenseSZ}{\renewcommand{\licenseAcknowledgement}{\textbf{Acknowledgements:} Stitz Zeager Open Source Mathematics (https://www.stitz-zeager.com/) }}
\newcommand{\licenseAPCSZ}{\renewcommand{\licenseAcknowledgement}{\textbf{Acknowledgements:} Active Prelude to Calculus (https://activecalculus.org/prelude) and Stitz Zeager Open Source Mathematics (https://www.stitz-zeager.com/) }}
\newcommand{\licenseORCCA}{\renewcommand{\licenseAcknowledgement}{\textbf{Acknowledgements:}Original source material, products with readable and accessible
math content, and other information freely available at pcc.edu/orcca.}}
\newcommand{\licenseY}{\renewcommand{\licenseAcknowledgement}{\textbf{Acknowledgements:} Yoshiwara Books (https://yoshiwarabooks.org/)}}
\newcommand{\licenseOS}{\renewcommand{\licenseAcknowledgement}{\textbf{Acknowledgements:} OpenStax College Algebra (https://openstax.org/details/books/college-algebra)}}
\newcommand{\licenseAPCSZCSCC}{\renewcommand{\licenseAcknowledgement}{\textbf{Acknowledgements:} Active Prelude to Calculus (https://activecalculus.org/prelude), Stitz Zeager Open Source Mathematics (https://www.stitz-zeager.com/), CSCC PreCalculus and Calculus texts (https://ximera.osu.edu/csccmathematics)}}

\ifxake\else %% do nothing on the website
\usepackage{fancyhdr}
\pagestyle{fancy}
\fancyhf{}
\fancyhead[R]{\sectionmark}
\fancyfoot[L]{\thepage}
\fancyfoot[C]{\licenseAcknowledgement}
\renewcommand{\headrulewidth}{0pt}
\renewcommand{\footrulewidth}{0pt}
\fi

%%%%%%%%%%%%%%%%



%\typeout{************************************************}
%\typeout{Table of Contents}
%\typeout{************************************************}


%% Edit this to change the font style
\newcommand{\sectionHeadStyle}{\sffamily\bfseries}


\makeatletter

%% part uses arabic numerals
\renewcommand*\thepart{\arabic{part}}


\ifxake\else
\renewcommand\chapterstyle{%
  \def\maketitle{%
    \addtocounter{titlenumber}{1}%
    \pagestyle{fancy}
    \phantomsection
    \addcontentsline{toc}{section}{\textbf{\thepart.\thetitlenumber\hspace{1em}\@title}}%
                    {\flushleft\small\sectionHeadStyle\@pretitle\par\vspace{-1.5em}}%
                    {\flushleft\LARGE\sectionHeadStyle\thepart.\thetitlenumber\hspace{1em}\@title \par }%
                    {\setcounter{problem}{0}\setcounter{sectiontitlenumber}{0}}%
                    \par}}





\renewcommand\sectionstyle{%
  \def\maketitle{%
    \addtocounter{sectiontitlenumber}{1}
    \pagestyle{fancy}
    \phantomsection
    \addcontentsline{toc}{subsection}{\thepart.\thetitlenumber.\thesectiontitlenumber\hspace{1em}\@title}%
    {\flushleft\small\sectionHeadStyle\@pretitle\par\vspace{-1.5em}}%
    {\flushleft\Large\sectionHeadStyle\thepart.\thetitlenumber.\thesectiontitlenumber\hspace{1em}\@title \par}%
    %{\setcounter{subsectiontitlenumber}{0}}%
    \par}}



\renewcommand\section{\@startsection{paragraph}{10}{\z@}%
                                     {-3.25ex\@plus -1ex \@minus -.2ex}%
                                     {1.5ex \@plus .2ex}%
                                     {\normalfont\large\sectionHeadStyle}}
\renewcommand\subsection{\@startsection{subparagraph}{10}{\z@}%
                                    {3.25ex \@plus1ex \@minus.2ex}%
                                    {-1em}%
                                    {\normalfont\normalsize\sectionHeadStyle}}

\fi

%% redefine Part
\renewcommand\part{%
   {\setcounter{titlenumber}{0}}
  \if@openright
    \cleardoublepage
  \else
    \clearpage
  \fi
  \thispagestyle{plain}%
  \if@twocolumn
    \onecolumn
    \@tempswatrue
  \else
    \@tempswafalse
  \fi
  \null\vfil
  \secdef\@part\@spart}

\def\@part[#1]#2{%
    \ifnum \c@secnumdepth >-2\relax
      \refstepcounter{part}%
      \addcontentsline{toc}{part}{\thepart\hspace{1em}#1}%
    \else
      \addcontentsline{toc}{part}{#1}%
    \fi
    \markboth{}{}%
    {\centering
     \interlinepenalty \@M
     \normalfont
     \ifnum \c@secnumdepth >-2\relax
       \huge\sffamily\bfseries \partname\nobreakspace\thepart
       \par
       \vskip 20\p@
     \fi
     \Huge \bfseries #2\par}%
    \@endpart}
\def\@spart#1{%
    {\centering
     \interlinepenalty \@M
     \normalfont
     \Huge \bfseries #1\par}%
    \@endpart}
\def\@endpart{\vfil\newpage
              \if@twoside
               \if@openright
                \null
                \thispagestyle{empty}%
                \newpage
               \fi
              \fi
              \if@tempswa
                \twocolumn
                \fi}



\makeatother





%\typeout{************************************************}
%\typeout{Stuff from Ximera}
%\typeout{************************************************}



\usepackage{array}  %% This is for typesetting long division
\setlength{\extrarowheight}{+.1cm}
\newdimen\digitwidth
\settowidth\digitwidth{9}
\def\divrule#1#2{
\noalign{\moveright#1\digitwidth
\vbox{\hrule width#2\digitwidth}}}





\newcommand{\RR}{\mathbb R}
\newcommand{\R}{\mathbb R}
\newcommand{\N}{\mathbb N}
\newcommand{\Z}{\mathbb Z}

\newcommand{\sagemath}{\textsf{SageMath}}


\def\d{\,d}
%\renewcommand{\d}{\mathop{}\!d}
\newcommand{\dd}[2][]{\frac{\d #1}{\d #2}}
\newcommand{\pp}[2][]{\frac{\partial #1}{\partial #2}}
\renewcommand{\l}{\ell}
\newcommand{\ddx}{\frac{d}{\d x}}



%\newcommand{\unit}{\,\mathrm}
\newcommand{\unit}{\mathop{}\!\mathrm}
\newcommand{\eval}[1]{\bigg[ #1 \bigg]}
\newcommand{\seq}[1]{\left( #1 \right)}
\renewcommand{\epsilon}{\varepsilon}
\renewcommand{\phi}{\varphi}


\renewcommand{\iff}{\Leftrightarrow}

\DeclareMathOperator{\arccot}{arccot}
\DeclareMathOperator{\arcsec}{arcsec}
\DeclareMathOperator{\arccsc}{arccsc}
\DeclareMathOperator{\sign}{sign}


%\DeclareMathOperator{\divergence}{divergence}
%\DeclareMathOperator{\curl}[1]{\grad\cross #1}
\newcommand{\lto}{\mathop{\longrightarrow\,}\limits}

\renewcommand{\bar}{\overline}

\colorlet{textColor}{black}
\colorlet{background}{white}
\colorlet{penColor}{blue!50!black} % Color of a curve in a plot
\colorlet{penColor2}{red!50!black}% Color of a curve in a plot
\colorlet{penColor3}{red!50!blue} % Color of a curve in a plot
\colorlet{penColor4}{green!50!black} % Color of a curve in a plot
\colorlet{penColor5}{orange!80!black} % Color of a curve in a plot
\colorlet{penColor6}{yellow!70!black} % Color of a curve in a plot
\colorlet{fill1}{penColor!20} % Color of fill in a plot
\colorlet{fill2}{penColor2!20} % Color of fill in a plot
\colorlet{fillp}{fill1} % Color of positive area
\colorlet{filln}{penColor2!20} % Color of negative area
\colorlet{fill3}{penColor3!20} % Fill
\colorlet{fill4}{penColor4!20} % Fill
\colorlet{fill5}{penColor5!20} % Fill
\colorlet{gridColor}{gray!50} % Color of grid in a plot

\newcommand{\surfaceColor}{violet}
\newcommand{\surfaceColorTwo}{redyellow}
\newcommand{\sliceColor}{greenyellow}




\pgfmathdeclarefunction{gauss}{2}{% gives gaussian
  \pgfmathparse{1/(#2*sqrt(2*pi))*exp(-((x-#1)^2)/(2*#2^2))}%
}





%\typeout{************************************************}
%\typeout{ORCCA Preamble.Tex}
%\typeout{************************************************}


%% \usepackage{geometry}
%% \geometry{letterpaper,total={408pt,9.0in}}
%% Custom Page Layout Adjustments (use latex.geometry)
%% \usepackage{amsmath,amssymb}
%% \usepackage{pgfplots}
\usepackage{pifont}                                         %needed for symbols, s.a. airplane symbol
\usetikzlibrary{positioning,fit,backgrounds}                %needed for nested diagrams
\usetikzlibrary{calc,trees,positioning,arrows,fit,shapes}   %needed for set diagrams
\usetikzlibrary{decorations.text}                           %needed for text following a curve
\usetikzlibrary{arrows,arrows.meta}                         %needed for open/closed intervals
\usetikzlibrary{positioning,3d,shapes.geometric}            %needed for 3d number sets tower

%% NEEDED FOR XIMERA 1
%\usetkzobj{all}       %NO LONGER VALID
%%%%%%%%%%%%%%

\usepackage{tikz-3dplot}
\usepackage{tkz-euclide}                     %needed for triangle diagrams
\usepgfplotslibrary{fillbetween}                            %shade regions of a plot
\usetikzlibrary{shadows}                                    %function diagrams
\usetikzlibrary{positioning}                                %function diagrams
\usetikzlibrary{shapes}                                     %function diagrams
%%% global colors from https://www.pcc.edu/web-services/style-guide/basics/color/ %%%
\definecolor{ruby}{HTML}{9E0C0F}
\definecolor{turquoise}{HTML}{008099}
\definecolor{emerald}{HTML}{1c8464}
\definecolor{amber}{HTML}{c7502a}
\definecolor{amethyst}{HTML}{70485b}
\definecolor{sapphire}{HTML}{263c53}
\colorlet{firstcolor}{sapphire}
\colorlet{secondcolor}{turquoise}
\colorlet{thirdcolor}{emerald}
\colorlet{fourthcolor}{amber}
\colorlet{fifthcolor}{amethyst}
\colorlet{sixthcolor}{ruby}
\colorlet{highlightcolor}{green!50!black}
\colorlet{graphbackground}{white}
\colorlet{wood}{brown!60!white}
%%% curve, dot, and graph custom styles %%%
\pgfplotsset{firstcurve/.style      = {color=firstcolor,  mark=none, line width=1pt, {Kite}-{Kite}, solid}}
\pgfplotsset{secondcurve/.style     = {color=secondcolor, mark=none, line width=1pt, {Kite}-{Kite}, solid}}
\pgfplotsset{thirdcurve/.style      = {color=thirdcolor,  mark=none, line width=1pt, {Kite}-{Kite}, solid}}
\pgfplotsset{fourthcurve/.style     = {color=fourthcolor, mark=none, line width=1pt, {Kite}-{Kite}, solid}}
\pgfplotsset{fifthcurve/.style      = {color=fifthcolor,  mark=none, line width=1pt, {Kite}-{Kite}, solid}}
\pgfplotsset{highlightcurve/.style  = {color=highlightcolor,  mark=none, line width=5pt, -, opacity=0.3}}   % thick, opaque curve for highlighting
\pgfplotsset{asymptote/.style       = {color=gray, mark=none, line width=1pt, <->, dashed}}
\pgfplotsset{symmetryaxis/.style    = {color=gray, mark=none, line width=1pt, <->, dashed}}
\pgfplotsset{guideline/.style       = {color=gray, mark=none, line width=1pt, -}}
\tikzset{guideline/.style           = {color=gray, mark=none, line width=1pt, -}}
\pgfplotsset{altitude/.style        = {dashed, color=gray, thick, mark=none, -}}
\tikzset{altitude/.style            = {dashed, color=gray, thick, mark=none, -}}
\pgfplotsset{radius/.style          = {dashed, thick, mark=none, -}}
\tikzset{radius/.style              = {dashed, thick, mark=none, -}}
\pgfplotsset{rightangle/.style      = {color=gray, mark=none, -}}
\tikzset{rightangle/.style          = {color=gray, mark=none, -}}
\pgfplotsset{closedboundary/.style  = {color=black, mark=none, line width=1pt, {Kite}-{Kite},solid}}
\tikzset{closedboundary/.style      = {color=black, mark=none, line width=1pt, {Kite}-{Kite},solid}}
\pgfplotsset{openboundary/.style    = {color=black, mark=none, line width=1pt, {Kite}-{Kite},dashed}}
\tikzset{openboundary/.style        = {color=black, mark=none, line width=1pt, {Kite}-{Kite},dashed}}
\tikzset{verticallinetest/.style    = {color=gray, mark=none, line width=1pt, <->,dashed}}
\pgfplotsset{soliddot/.style        = {color=firstcolor,  mark=*, only marks}}
\pgfplotsset{hollowdot/.style       = {color=firstcolor,  mark=*, only marks, fill=graphbackground}}
\pgfplotsset{blankgraph/.style      = {xmin=-10, xmax=10,
                                        ymin=-10, ymax=10,
                                        axis line style={-, draw opacity=0 },
                                        axis lines=box,
                                        major tick length=0mm,
                                        xtick={-10,-9,...,10},
                                        ytick={-10,-9,...,10},
                                        grid=major,
                                        grid style={solid,gray!20},
                                        xticklabels={,,},
                                        yticklabels={,,},
                                        minor xtick=,
                                        minor ytick=,
                                        xlabel={},ylabel={},
                                        width=0.75\textwidth,
                                      }
            }
\pgfplotsset{numberline/.style      = {xmin=-10,xmax=10,
                                        minor xtick={-11,-10,...,11},
                                        xtick={-10,-5,...,10},
                                        every tick/.append style={thick},
                                        axis y line=none,
                                        y=15pt,
                                        axis lines=middle,
                                        enlarge x limits,
                                        grid=none,
                                        clip=false,
                                        axis background/.style={},
                                        after end axis/.code={
                                          \path (axis cs:0,0)
                                          node [anchor=north,yshift=-0.075cm] {\footnotesize 0};
                                        },
                                        every axis x label/.style={at={(current axis.right of origin)},anchor=north},
                                      }
            }
\pgfplotsset{openinterval/.style={color=firstcolor,mark=none,ultra thick,{Parenthesis}-{Parenthesis}}}
\pgfplotsset{openclosedinterval/.style={color=firstcolor,mark=none,ultra thick,{Parenthesis}-{Bracket}}}
\pgfplotsset{closedinterval/.style={color=firstcolor,mark=none,ultra thick,{Bracket}-{Bracket}}}
\pgfplotsset{closedopeninterval/.style={color=firstcolor,mark=none,ultra thick,{Bracket}-{Parenthesis}}}
\pgfplotsset{infiniteopeninterval/.style={color=firstcolor,mark=none,ultra thick,{Kite}-{Parenthesis}}}
\pgfplotsset{openinfiniteinterval/.style={color=firstcolor,mark=none,ultra thick,{Parenthesis}-{Kite}}}
\pgfplotsset{infiniteclosedinterval/.style={color=firstcolor,mark=none,ultra thick,{Kite}-{Bracket}}}
\pgfplotsset{closedinfiniteinterval/.style={color=firstcolor,mark=none,ultra thick,{Bracket}-{Kite}}}
\pgfplotsset{infiniteinterval/.style={color=firstcolor,mark=none,ultra thick,{Kite}-{Kite}}}
\pgfplotsset{interval/.style= {ultra thick, -}}
%%% cycle list of plot styles for graphs with multiple plots %%%
\pgfplotscreateplotcyclelist{pccstylelist}{%
  firstcurve\\%
  secondcurve\\%
  thirdcurve\\%
  fourthcurve\\%
  fifthcurve\\%
}
%%% default plot settings %%%
\pgfplotsset{every axis/.append style={
  axis x line=middle,    % put the x axis in the middle
  axis y line=middle,    % put the y axis in the middle
  axis line style={<->}, % arrows on the axis
  scaled ticks=false,
  tick label style={/pgf/number format/fixed},
  xlabel={$x$},          % default put x on x-axis
  ylabel={$y$},          % default put y on y-axis
  xmin = -7,xmax = 7,    % most graphs have this window
  ymin = -7,ymax = 7,    % most graphs have this window
  domain = -7:7,
  xtick = {-6,-4,...,6}, % label these ticks
  ytick = {-6,-4,...,6}, % label these ticks
  yticklabel style={inner sep=0.333ex},
  minor xtick = {-7,-6,...,7}, % include these ticks, some without label
  minor ytick = {-7,-6,...,7}, % include these ticks, some without label
  scale only axis,       % don't consider axis and tick labels for width and height calculation
  cycle list name=pccstylelist,
  tick label style={font=\footnotesize},
  legend cell align=left,
  grid = both,
  grid style = {solid,gray!20},
  axis background/.style={fill=graphbackground},
}}
\pgfplotsset{framed/.style={axis background/.style ={draw=gray}}}
%\pgfplotsset{framed/.style={axis background/.style ={draw=gray,fill=graphbackground,rounded corners=3ex}}}
%%% other tikz (not pgfplots) settings %%%
%\tikzset{axisnode/.style={font=\scriptsize,text=black}}
\tikzset{>=stealth}
%%% for nested diagram in types of numbers section %%%
\newcommand\drawnestedsets[4]{
  \def\position{#1}             % initial position
  \def\nbsets{#2}               % number of sets
  \def\listofnestedsets{#3}     % list of sets
  \def\reversedlistofcolors{#4} % reversed list of colors
  % position and draw labels of sets
  \coordinate (circle-0) at (#1);
  \coordinate (set-0) at (#1);
  \foreach \set [count=\c] in \listofnestedsets {
    \pgfmathtruncatemacro{\cminusone}{\c - 1}
    % label of current set (below previous nested set)
    \node[below=3pt of circle-\cminusone,inner sep=0]
    (set-\c) {\set};
    % current set (fit current label and previous set)
    \node[circle,inner sep=0,fit=(circle-\cminusone)(set-\c)]
    (circle-\c) {};
  }
  % draw and fill sets in reverse order
  \begin{scope}[on background layer]
    \foreach \col[count=\c] in \reversedlistofcolors {
      \pgfmathtruncatemacro{\invc}{\nbsets-\c}
      \pgfmathtruncatemacro{\invcplusone}{\invc+1}
      \node[circle,draw,fill=\col,inner sep=0,
      fit=(circle-\invc)(set-\invcplusone)] {};
    }
  \end{scope}
  }
\ifdefined\tikzset
\tikzset{ampersand replacement = \amp}
\fi
\newcommand{\abs}[1]{\left\lvert#1\right\rvert}
%\newcommand{\point}[2]{\left(#1,#2\right)}
\newcommand{\highlight}[1]{\definecolor{sapphire}{RGB}{59,90,125} {\color{sapphire}{{#1}}}}
\newcommand{\firsthighlight}[1]{\definecolor{sapphire}{RGB}{59,90,125} {\color{sapphire}{{#1}}}}
\newcommand{\secondhighlight}[1]{\definecolor{emerald}{RGB}{20,97,75} {\color{emerald}{{#1}}}}
\newcommand{\unhighlight}[1]{{\color{black}{{#1}}}}
\newcommand{\lowlight}[1]{{\color{lightgray}{#1}}}
\newcommand{\attention}[1]{\mathord{\overset{\downarrow}{#1}}}
\newcommand{\nextoperation}[1]{\mathord{\boxed{#1}}}
\newcommand{\substitute}[1]{{\color{blue}{{#1}}}}
\newcommand{\pinover}[2]{\overset{\overset{\mathrm{\ #2\ }}{|}}{\strut #1 \strut}}
\newcommand{\addright}[1]{{\color{blue}{{{}+#1}}}}
\newcommand{\addleft}[1]{{\color{blue}{{#1+{}}}}}
\newcommand{\subtractright}[1]{{\color{blue}{{{}-#1}}}}
\newcommand{\multiplyright}[2][\cdot]{{\color{blue}{{{}#1#2}}}}
\newcommand{\multiplyleft}[2][\cdot]{{\color{blue}{{#2#1{}}}}}
\newcommand{\divideunder}[2]{\frac{#1}{{\color{blue}{{#2}}}}}
\newcommand{\divideright}[1]{{\color{blue}{{{}\div#1}}}}
\newcommand{\negate}[1]{{\color{blue}{{-}}}\left(#1\right)}
\newcommand{\cancelhighlight}[1]{\definecolor{sapphire}{RGB}{59,90,125}{\color{sapphire}{{\cancel{#1}}}}}
\newcommand{\secondcancelhighlight}[1]{\definecolor{emerald}{RGB}{20,97,75}{\color{emerald}{{\bcancel{#1}}}}}
\newcommand{\thirdcancelhighlight}[1]{\definecolor{amethyst}{HTML}{70485b}{\color{amethyst}{{\xcancel{#1}}}}}
\newcommand{\lt}{<} %% Bart: WHY?
\newcommand{\gt}{>} %% Bart: WHY?
\newcommand{\amp}{&} %% Bart: WHY?


%%% These commands break Xake
%% \newcommand{\apple}{\text{🍎}}
%% \newcommand{\banana}{\text{🍌}}
%% \newcommand{\pear}{\text{🍐}}
%% \newcommand{\cat}{\text{🐱}}
%% \newcommand{\dog}{\text{🐶}}

\newcommand{\apple}{PICTURE OF APPLE}
\newcommand{\banana}{PICTURE OF BANANA}
\newcommand{\pear}{PICTURE OF PEAR}
\newcommand{\cat}{PICTURE OF CAT}
\newcommand{\dog}{PICTURE OF DOG}


%%%%% INDEX STUFF
\newcommand{\dfn}[1]{\textbf{#1}\index{#1}}
\usepackage{imakeidx}
\makeindex[intoc]
\makeatletter
\gdef\ttl@savemark{\sectionmark{}}
\makeatother












 % for drawing cube in Optimization problem
\usetikzlibrary{quotes,arrows.meta}
\tikzset{
  annotated cuboid/.pic={
    \tikzset{%
      every edge quotes/.append style={midway, auto},
      /cuboid/.cd,
      #1
    }
    \draw [every edge/.append style={pic actions, densely dashed, opacity=.5}, pic actions]
    (0,0,0) coordinate (o) -- ++(-\cubescale*\cubex,0,0) coordinate (a) -- ++(0,-\cubescale*\cubey,0) coordinate (b) edge coordinate [pos=1] (g) ++(0,0,-\cubescale*\cubez)  -- ++(\cubescale*\cubex,0,0) coordinate (c) -- cycle
    (o) -- ++(0,0,-\cubescale*\cubez) coordinate (d) -- ++(0,-\cubescale*\cubey,0) coordinate (e) edge (g) -- (c) -- cycle
    (o) -- (a) -- ++(0,0,-\cubescale*\cubez) coordinate (f) edge (g) -- (d) -- cycle;
    \path [every edge/.append style={pic actions, |-|}]
    (b) +(0,-5pt) coordinate (b1) edge ["x"'] (b1 -| c)
    (b) +(-5pt,0) coordinate (b2) edge ["y"] (b2 |- a)
    (c) +(3.5pt,-3.5pt) coordinate (c2) edge ["x"'] ([xshift=3.5pt,yshift=-3.5pt]e)
    ;
  },
  /cuboid/.search also={/tikz},
  /cuboid/.cd,
  width/.store in=\cubex,
  height/.store in=\cubey,
  depth/.store in=\cubez,
  units/.store in=\cubeunits,
  scale/.store in=\cubescale,
  width=10,
  height=10,
  depth=10,
  units=cm,
  scale=.1,
}

\author{Elizabeth Campolongo}
\license{Creative Commons Attribution-ShareAlike 4.0 International License}
\acknowledgement{https://www.stitz-zeager.com/szprecalculus07042013.pdf}

\title{Algebra of Secant Lines}

\begin{document}
\begin{abstract}
  
\end{abstract}
\maketitle


%\typeout{************************************************}
%\typeout{Motivating Questions}
%\typeout{************************************************}

\begin{motivatingQuestions}\begin{itemize}
\item What are some algebra techniques that allow us to simplify the equation of a secant line?
\item Why is this important?
\end{itemize}\end{motivatingQuestions}


%\typeout{************************************************}
%\typeout{Subsection Introduction}
%\typeout{************************************************}

\section{Introduction}

Given the graph of a function $y = f(x)$, we have discussed methods to determine the slope of the secant line between two points, $(a,f(a))$ and $(b,f(b))$, on the graph. We know that this slope represents the average rate of change of the function $f$ on the interval $[a,b]$, denoted by $AV_{[a,b]}$. Both of these can be rewritten by letting $b = h-a$, so that we have the value $h$ representing the horizontal distance between the points. This means that as $h\rightarrow 0$, the secant line, or the average rate of change of the function, approaches a value known as the slope of the tangent line of $f$ at $a$. This will be discussed extensively in future calculus courses, but in this section we will focus on tools to simplify the expression $AV_{[a,a+h]}$, as they are essential to calculating this limit.

%We have discussed secant lines to the graph of a function $y=f(x)$. If $(a,f(a))$ and $(b,f(b))$ are two such points, the slope $$m=\frac{f(b)-f(a)}{b-a}$$of the secant line passing through such points is, in fact, the average rate of change of $f$ on the interval $[a,b]$. It is also convenient to let $h=b-a$, so the slope expression becomes $$m=\frac{f(a+h)-f(a)}{h},$$which is really equivalent to the previous one.
%
%\begin{image}[2in]
%		  \begin{tikzpicture}
%		  \draw [thick, scale = 3, domain = -.5:2, smooth, variable =\x] plot ({\x},  {3/4*exp(\x)*cos(deg(\x)) + 1/2*exp(\x)*sin(deg(\x))});%{\x*(3-2*\x*\x)});
%		    \coordinate (C) at (0,2.8);
%		    \coordinate (D) at (5,1);
%		    \coordinate (E) at (5,7.5);
%		   % \draw[decoration={brace,mirror,raise=.2cm},decorate,thin] (0,2)--(4,2);
%		   % \draw[decoration={brace,mirror,raise=.2cm},decorate,thin] (4,2)--(4,6.5);
%		  %  \draw[decoration={brace,raise=.2cm},decorate,thin] (0,2)--(4,6.5);
%		    \draw[blue] (C)--(E)--cycle;
%		 %   \draw[dotted] (C) edge (D);
%		%    \draw[dotted] (E) edge (D);
%		      \node at (2,7) {$y= f(x)$};
%		   \node at (.4,4) {$(a, f(a))$};
%		   % \node[rotate=-90] at (4+.5,4) {A'};
%		 %   \node[rotate=35] at (.8,4.8) {$f(x)-f(x_1) = m(x-x_1)$};
%		    \node at (5.1,7.8) {$(a+h,f(a+h))$};
%		%    \node at (-4.7, 1) {$(x_1,y_1)$};
%		  \end{tikzpicture}
%		\end{image}
%

%\begin{image}[2in]
%		  \begin{tikzpicture}
%		  \draw [thick, scale = 3, domain = -.5:2, smooth, variable =\x] plot ({\x},  {3/4*exp(\x)*cos(deg(\x)) + 1/2*exp(\x)*sin(deg(\x))});%{\x*(3-2*\x*\x)});
%		    \coordinate (C) at (0,2.8);
%		    \coordinate (D) at (5,1);
%		    \coordinate (E) at (5,7.5);
%		    \coordinate (A) at (0,2);
%		    \coordinate (B) at (2.4,6);
%		   % \draw[decoration={brace,mirror,raise=.2cm},decorate,thin] (0,2)--(4,2);
%		   % \draw[decoration={brace,mirror,raise=.2cm},decorate,thin] (4,2)--(4,6.5);
%		  %  \draw[decoration={brace,raise=.2cm},decorate,thin] (0,2)--(4,6.5);
%		    \draw[blue] (C)--(E)--cycle;
%		    \draw[red] (A)--(B)--cycle;
%		 %   \draw[dotted] (C) edge (D);
%		%    \draw[dotted] (E) edge (D);
%		      \node at (6.4,5.6) {$y= f(x)$};
%		   \node at (.4,4) {$(a, f(a))$};
%		   % \node[rotate=-90] at (4+.5,4) {A'};
%		 %   \node[rotate=35] at (.8,4.8) {$f(x)-f(x_1) = m(x-x_1)$};
%		    \node at (5.1,7.8) {$(a+h,f(a+h))$};
%		%    \node at (-4.7, 1) {$(x_1,y_1)$};
%		  \end{tikzpicture}
%		\end{image}

%tangent line parallel to secant line
%\begin{image}[2in]
%		  \begin{tikzpicture}
%		  \draw [thick, scale = 3, domain = -.5:2, smooth, variable =\x] plot ({\x},  {3/4*exp(\x)*cos(deg(\x)) + 1/2*exp(\x)*sin(deg(\x))});%{\x*(3-2*\x*\x)});
%		    \coordinate (C) at (0,2.8);
%		    \coordinate (D) at (5,1);
%		    \coordinate (E) at (5,7.5);
%		    \coordinate (A) at (1.5,5.41);
%		    \coordinate (B) at (4.5,8.23);
%		   % \draw[decoration={brace,mirror,raise=.2cm},decorate,thin] (0,2)--(4,2);
%		   % \draw[decoration={brace,mirror,raise=.2cm},decorate,thin] (4,2)--(4,6.5);
%		  %  \draw[decoration={brace,raise=.2cm},decorate,thin] (0,2)--(4,6.5);
%		    \draw[blue] (C)--(E)--cycle;
%		    \draw[red] (A)--(B)--cycle;
%		 %   \draw[dotted] (C) edge (D);
%		%    \draw[dotted] (E) edge (D);
%		      \node at (6.4,5.6) {$y= f(x)$};
%		   \node at (.4,4) {$(a, f(a))$};
%		   % \node[rotate=-90] at (4+.5,4) {A'};
%		 %   \node[rotate=35] at (.8,4.8) {$f(x)-f(x_1) = m(x-x_1)$};
%		    \node at (5.1,7.8) {$(a+h,f(a+h))$};
%		%    \node at (-4.7, 1) {$(x_1,y_1)$};
%		  \end{tikzpicture}
%		\end{image}




\section{Definitions and examples}
%relate to difference of squares: rethinking difference of squares through square roots
%simplfying polynomials and fractions!

Recall the special formula for difference of squares, $a^2-b^2 = (a-b)(a+b)$. For non-square values of $a$ and $b$ we can use the same idea to rationalize differences (or sums) of square roots through multiplication by the corresponding sum (or difference), which we call the {\it conjugate}.
Given any expression $\sqrt{a} \pm \sqrt{b}$, $a,b$ real numbers, the conjugate of this expression is $\sqrt{a} \mp \sqrt{b}$. Multiplying such an expression by its conjugate rationalizes it through the distributive property: $(\sqrt{a} + \sqrt{b})\cdot(\sqrt{a} - \sqrt{b}) = (\sqrt{a})^2 + \sqrt{a}\sqrt{b} - \sqrt{a}\sqrt{b} - (\sqrt{b})^2 = a - b$

\begin{callout}
  {\bf Definition:} 
  Given any difference of positive values $a-b$, we know from the difference of squares, that $a-b = (\sqrt{a}-\sqrt{b})(\sqrt{a}+\sqrt{b})$. \\
  The sum $\sqrt{a}+\sqrt{b}$ is the {\it conjugate} of the difference $\sqrt{a}-\sqrt{b}$.
  Likewise, the difference $\sqrt{a}-\sqrt{b}$ is the {\it conjugate} of the sum $\sqrt{a}+\sqrt{b}$.
 
 Multiplying such an expression by its conjugate will rationalize the expression.
%Given an expression of the form $\frac{\sqrt{x+h} - \sqrt{x}}{h}$, we can multiply by the {\bf conjugate} of the numerator to rationalize it. The conjugate of the numerator is $\sqrt{x+h}+\sqrt{x}$. 
  \end{callout}
%
Note that this is one of the most important tools in your simplification toolbox. Other tools include simplifying polynomials and fractions (finding the common denominator), moving coefficients inside or outside the square root, and the trigonometric identities introduced in Section 10-2.

%[moving coefficient inside or outside square root, absolute value functions]
%
\begin{example}
For the following, find the difference quotient. Simplify as much as possible
 %
  \begin{enumerate}
  \item $f(x) = \sqrt{x}$, $x \geq 0$

    \begin{explanation}
    We consider $h>0$ to avoid any potential undefined values plugged into our function $f$ since its domain is $[0,\infty)$.
    \begin{equation}\label{sqrtDQ}
     \frac{f(x+h) - f(x)}{h} = \frac{\sqrt{x+h} - \sqrt{x}}{h}
    \end{equation}
    %
    Observe that we cannot combine any terms in \eqref{sqrtDQ}, but the numerator is of the form $\sqrt{a} - \sqrt{b}$. Hence, we will multiply by the conjugate to rationalize the numerator:
   \begin{align}
   \frac{\sqrt{x+h} - \sqrt{x}}{h} & =  \frac{(\sqrt{x+h} - \sqrt{x})}{h} \cdot  \frac{(\sqrt{x+h} + \sqrt{x})}{(\sqrt{x+h} + \sqrt{x})} \label{multbyC} \\
   &=  \frac{(\sqrt{x+h})^2 - (\sqrt{x})^2}{h(\sqrt{x+h}+\sqrt{x})}\label{diffSq}
   \end{align}
   %
   Remember, in \eqref{multbyC}, that in order to avoid changing the value of the expression, we must multiply by the conjugate over itself, i.e., multiply by 1. Then \eqref{diffSq} has a difference of squares in the numerator and is equal to 
   %
   \begin{align*}
    \frac{(x+h) - x}{h(\sqrt{x+h} + \sqrt{x})} & = \frac{h}{h(\sqrt{x+h}+\sqrt{x})} \\
    &=\frac{1}{\sqrt{x+h}+\sqrt{x}},
   \end{align*} 
   by cancelling out the $h$ in the numerator and the denominator. 
   
   This expression that we have found is now in a form that allows us to consider what happens when $h \rightarrow 0$ by removing the $h$ from the denominator. 
       \end{explanation}
       %
  \item $g(x) = \sqrt{8-2x}$, $x \geq 4$.\\
  \begin{explanation}
  We consider $h>0$ to avoid any potential undefined values plugged into our function $g$ since its domain is $[4,\infty)$.

  \begin{equation}\label{sqDQ}
   \frac{g(x+h) - g(x)}{h} = \frac{\sqrt{8-4(x+h)} - \sqrt{8-4x}}{h}
  \end{equation}
  %
  Once again, we cannot combine any terms in the numerator of \eqref{sqDQ}, so we will multiply by the conjugate to rationalize the numerator, hoping we will be able to simplify the equation. \eqref{sqDQ} is equal to
  %
  \begin{align*}
 % \frac{\sqrt{8-2(x+h)} - \sqrt{8-2x}}{h} &=
 \frac{(\sqrt{8-4(x+h)} - \sqrt{8-4x})}{h} &\cdot \frac{(\sqrt{8-4(x+h)} + \sqrt{8-4x})}{(\sqrt{8-4(x+h)} + \sqrt{8-4x})} \\
  &=  \frac{(\sqrt{8-4(x+h)})^2 - (\sqrt{8-4x})^2}{h(\sqrt{8-4(x+h)} + \sqrt{8-4x})} \\
  &= \frac{(8-4x-4h) - (8-4x)}{h(\sqrt{4(2-(x+h))} + \sqrt{4(2-x)})}\\
  &= \frac{-4h}{2h(\sqrt{2-(x+h)} + \sqrt{2-x})}
  \end{align*}
  %
  Now, we simply cancel the $2h$ in the numerator and the denominator, giving
  $$ \frac{g(x+h) - g(x)}{h} = \frac{-2}{\sqrt{2-(x+h)} + \sqrt{2-x}}.$$
  \end{explanation}
  %
  \item $f(x) = \cos(2x)$
 \\
 \begin{explanation}
 Note that $\cos(z)$ is defined for all real numbers $z$, so we need not worry about the values of $x$ and $h$ plugged into the difference quotient formula.
 \begin{equation*}\label{cosDQ}
 \frac{f(x+h) - f(x)}{h} = \frac{\cos(2(x+h)) - \cos(2x)}{h}
 \end{equation*}
 Now, we can expand out using the summation formula for cosine:
 %
 \begin{align*}
 \frac{\cos(2x)\cos(2h)-\sin(2x)\sin(2h) - \cos(2x)}{h} 
 &= \frac{\cos(2x)(\cos(2h)-1)- \sin(2x)\sin(2h)}{h} \\
 &=  \cos(2x)\frac{\cos(2h)-1}{h}- \sin(2x)\frac{\sin(2h)}{h}
 \end{align*}
 
Here we can plug in decreasing values of $h$ for cosine and sine and start to notice a pattern..\\
Explore this further by changing the $x$ and $h$ values in the following Desmos graph:
\desmos{lf9wkgurwz}{800}{600}
 \end{explanation}
 %
 %
  \item $f(x) = |x-1|$ %for $|x|>1|$. 
  \\
  %
   \begin{explanation}
   We will consider two regions and ranges of $h$: (1) $x \in (-\infty,1)$ with $h<0$ and (2) $x\in(1, \infty)$ with $h>0$. 
   %Additionally, we will consider the case for $h$ positive and $h$ negative each separately. 
   
  Let's start with region (1), where $x<1$ and $h<0$. From this, we know that $x+h-1<x-1<0$, so $|x-1|=-(x-1)$. Hence we have
  \begin{align*}
 \frac{f(x+h) - f(x)}{h} &= \frac{|x+h-1| - |x-1|}{h}\\
 &= \frac{-x-h+1 + (x-1)}{h}\\
 &= \frac{-h}{h} = -1
 \end{align*}
 
 Alternatively, if we consider region (2), where $x>1$ and $h>0$, then we have $x+h-1>x-1>0$, so that
  \begin{align*}
 \frac{|x+h+1| - |x+1|}{h} &= \frac{x+h-1 - (x-1)}{h}\\
 &= \frac{h}{h} = 1
 \end{align*}
 
 Notice that this is not as clear-cut if we consider say $x<1$ and $h>0$. Then we would need to consider if $h$ is large enough that $x+h>1$. Let's explore this some more.
 
 Let $x<1$ and $h>0$. Further, assume $h>1-x$, then 
  \begin{align*}
 \frac{f(x+h) - f(x)}{h} &= \frac{|x+h-1| - |x-1|}{h}\\
 &= \frac{x+h+1 + (x-1)}{h}\\
 &= \frac{2x + h}{h}
 \end{align*}
Alternatively, if $h<1-x$, then
 \begin{align*}
 \frac{f(x+h) - f(x)}{h} &= \frac{|x+h-1| - |x-1|}{h}\\
 &= \frac{-x-h+1 + (x-1)}{h}\\
 &= \frac{-h}{h} = -1
 \end{align*}
As when $x<1$ and $h<0$.
 \end{explanation}
 
 \item $g(x) =  \frac{2x}{x^2 + 3}$\\
 \begin{explanation}
 First note that the denominator of our function $g$ is greater than zero for all real values of $x$, so the function is defined for all real numbers. Thus, we may calculate the difference quotient without concern for input values of $x$ and $h$.
 %
 \begin{equation*}
 \frac{g(x+h) - g(x)}{h} = \frac{\frac{2(x+h)}{(x+h)^2 + 3} - \frac{2x}{x^2+3}}{h}
 \end{equation*}
 %
 This expression for the difference quotient looks rather messy, so let's find the common denominator and see if we can cancel out some terms in the numerator by combining the fractions. We will leave the terms in the denominator in their current format, but multiply out the $(x+h)^2$ in the numerator for ease of simplification. 
 
 Note that the common denominator is $((x+h)^2+3)(x^2 + 3)$. Then we have
 \begin{align*}
  \frac{g(x+h) - g(x)}{h} &= \frac{\frac{(2x+2h)(x^2+3) - 2x(x^2+2xh+h^2+3)}{((x+h)^2+3)(x^2 + 3)}}{h} \\
  &= \frac{2x^3 +2x^2h + 6x + 6h - (2x^3 +4x^2h + 2xh^2 +6x)}{h\big((x+h)^2+3\big)(x^2 + 3)}
 \end{align*}
 Observe that we have combined the denominators now and have many common terms in the numerator that can be subtracted from each other, so that
 %
 \begin{equation*}
 \frac{g(x+h) - g(x)}{h} = \frac{-2x^2h + 6h -2xh^2}{h\big((x+h)^2+3\big)(x^2 + 3)}.
\end{equation*}
%
Now, all the terms in the numerator have a factor of $h$, so we can cancel the $h$ in the numerator and denominator for a final, simplified difference quotient of
 \begin{equation*}
\frac{-2x^2-2xh + 6}{\big((x+h)^2+3\big)(x^2 + 3)}.
\end{equation*}
 \end{explanation}
  \end{enumerate}
\end{example}



%\typeout{************************************************}
%\typeout{Summary}
%\typeout{************************************************}

\begin{summary}
Useful tools for simplification:
\begin{itemize}
\item Simplifying polynomials.
\item Simplifying fractions by finding common denominators.
\item Multiplying by the conjugate to rationalize the numerator.
\item Considering regions for absolute value functions. 
\end{itemize}\end{summary}




\end{document}
