%% Custom Preamble Entries, early (use latex.preamble.early)
%% Customized to load Palatino fonts
\usepackage[T1]{fontenc}
\renewcommand{\rmdefault}{zpltlf} %Roman font for use in math mode
\usepackage[scaled=.85]{beramono}% used only by \mathtt
\usepackage[type1]{cabin}%used only by \mathsf
\usepackage{amsmath,amssymb,amsthm}%load before newpxmath
\usepackage[varg,cmintegrals,bigdelims,varbb]{newpxmath}
\usepackage[scr=rsfso]{mathalfa}
\usepackage{bm} %load after all math to give access to bold math
% Now load the otf text fonts using fontspec--wont affect math
%\usepackage[no-math]{fontspec}
%\setmainfont{TeXGyrePagellaX}
%\defaultfontfeatures{Ligatures=TeX,Scale=1,Mapping=tex-text}
\linespread{1.02}

%% Default LaTeX packages
%%   1.  always employed (or nearly so) for some purpose, or
%%   2.  a stylewriter may assume their presence
\usepackage{geometry}
%% Some aspects of the preamble are conditional,
%% the LaTeX engine is one such determinant
\usepackage{ifthen}
%% etoolbox has a variety of modern conveniences
\usepackage{etoolbox}
\usepackage{ifxetex,ifluatex}
%% Raster graphics inclusion
\usepackage{graphicx}
%% Color support, xcolor package
%% Always loaded, for: add/delete text, author tools
%% Here, since tcolorbox loads tikz, and tikz loads xcolor
\PassOptionsToPackage{usenames,dvipsnames,svgnames,table}{xcolor}
\usepackage{xcolor}
%% Colored boxes, and much more, though mostly styling
%% skins library provides "enhanced" skin, employing tikzpicture
%% boxes may be configured as "breakable" or "unbreakable"
%% "raster" controls grids of boxes, aka side-by-side
\usepackage{tcolorbox}
\tcbuselibrary{skins}
\tcbuselibrary{breakable}
\tcbuselibrary{raster}
%% We load some "stock" tcolorbox styles that we use a lot
%% Placement here is provisional, there will be some color work also
%% First, black on white, no border, transparent, but no assumption about titles
\tcbset{ bwminimalstyle/.style={size=minimal, boxrule=-0.3pt, frame empty,
colback=white, colbacktitle=white, coltitle=black, opacityfill=0.0} }
%% Second, bold title, run-in to text/paragraph/heading
%% Space afterwards will be controlled by environment,
%% dependent of constructions of the tcb title
\tcbset{ runintitlestyle/.style={fonttitle=\normalfont\bfseries, attach title to upper} }
%% Spacing prior to each exercise, anywhere
\tcbset{ exercisespacingstyle/.style={before skip={1.5ex plus 0.5ex}} }
%% Spacing prior to each block
\tcbset{ blockspacingstyle/.style={before skip={2.0ex plus 0.5ex}} }
%% xparse allows the construction of more robust commands,
%% this is a necessity for isolating styling and behavior
%% The tcolorbox library of the same name loads the base library
\tcbuselibrary{xparse}
%% Hyperref should be here, but likes to be loaded late
%%
%% Inline math delimiters, \(, \), need to be robust
%% 2016-01-31:  latexrelease.sty  supersedes  fixltx2e.sty
%% If  latexrelease.sty  exists, bugfix is in kernel
%% If not, bugfix is in  fixltx2e.sty
%% See:  https://tug.org/TUGboat/tb36-3/tb114ltnews22.pdf
%% and read "Fewer fragile commands" in distribution's  latexchanges.pdf
\IfFileExists{latexrelease.sty}{}{\usepackage{fixltx2e}}
%% Text height identically 9 inches, text width varies on point size
%% See Bringhurst 2.1.1 on measure for recommendations
%% 75 characters per line (count spaces, punctuation) is target
%% which is the upper limit of Bringhurst's recommendations
\geometry{letterpaper,total={340pt,9.0in}}
%% Custom Page Layout Adjustments (use latex.geometry)
\geometry{paperwidth=7.44in,paperheight=9.69in,tmargin=.5in,bmargin=.3in,hmargin=.75in,bindingoffset=.4in,includeheadfoot }
%% This LaTeX file may be compiled with pdflatex, xelatex, or lualatex
%% The following provides engine-specific capabilities
%% Generally, xelatex and lualatex will do better languages other than US English
%% You can pick from the conditional if you will only ever use one engine
\ifthenelse{\boolean{xetex} \or \boolean{luatex}}{%
%% begin: xelatex and lualatex-specific configuration
%% fontspec package will make Latin Modern (lmodern) the default font
\ifxetex\usepackage{xltxtra}\fi
%\usepackage{fontspec}
%% realscripts is the only part of xltxtra relevant to lualatex 
\ifluatex\usepackage{realscripts}\fi
%% 
%% Extensive support for other languages
\usepackage{polyglossia}
%% Main document language is US English
\setdefaultlanguage{english}
%% Spanish
\setotherlanguage{spanish}
%% Vietnamese
\setotherlanguage{vietnamese}
%% end: xelatex and lualatex-specific configuration
}{%
%% begin: pdflatex-specific configuration
%% translate common Unicode to their LaTeX equivalents
%% Also, fontenc with T1 makes CM-Super the default font
%% (\input{ix-utf8enc.dfu} from the "inputenx" package is possible addition (broken?)
\usepackage[T1]{fontenc}
\usepackage[utf8]{inputenc}
%% end: pdflatex-specific configuration
}
%% Monospace font: Inconsolata (zi4)
%% Sponsored by TUG: http://levien.com/type/myfonts/inconsolata.html
%% See package documentation for excellent instructions
%% One caveat, seem to need full file name to locate OTF files
%% Loads the "upquote" package as needed, so we don't have to
%% Upright quotes might come from the  textcomp  package, which we also use
%% We employ the shapely \ell to match Google Font version
%% pdflatex: "varqu" option produces best upright quotes
%% xelatex,lualatex: add StylisticSet 1 for shapely \ell
%% xelatex,lualatex: add StylisticSet 2 for plain zero
%% xelatex,lualatex: we add StylisticSet 3 for upright quotes
%% 
\ifthenelse{\boolean{xetex} \or \boolean{luatex}}{%
%% begin: xelatex and lualatex-specific monospace font
\usepackage{zi4}
\setmonofont[BoldFont=Inconsolatazi4-Bold.otf,StylisticSet={1,3}]{Inconsolatazi4-Regular.otf}
%% end: xelatex and lualatex-specific monospace font
}{%
%% begin: pdflatex-specific monospace font
%% "varqu" option provides textcomp \textquotedbl glyph
%% "varl"  option provides shapely "ell"
\usepackage[varqu,varl]{zi4}
%% end: pdflatex-specific monospace font
}
%% \mono macro for content of "c" element, and XML parts
\newcommand{\mono}[1]{\texttt{#1}}
%% Symbols, align environment, bracket-matrix
\usepackage{amsmath}
\usepackage{amssymb}
%% allow page breaks within display mathematics anywhere
%% level 4 is maximally permissive
%% this is exactly the opposite of AMSmath package philosophy
%% there are per-display, and per-equation options to control this
%% split, aligned, gathered, and alignedat are not affected
\allowdisplaybreaks[4]
%% allow more columns to a matrix
%% can make this even bigger by overriding with  latex.preamble.late  processing option
\setcounter{MaxMatrixCols}{30}
%%
%%
%% Division Titles, and Page Headers/Footers
%% titlesec package, loading "titleps" package cooperatively
%% See code comments about the necessity and purpose of "explicit" option
%\usepackage[explicit, pagestyles]{titlesec}
%\newtitlemark{\chaptertitlename}
%\renewpagestyle{headings}{\sethead[\textsl{\ifthechapter{\chaptertitlename{} \thechapter }{} \chaptertitle}][][]
%{}{}{\textsl{\thesection{} \sectiontitle}}
%\setfoot[\thepage][][]
%{}{}{\thepage}}

%% Set global/default page style for document due
%% to potential re-definitions after documentclass
\pagestyle{headings}
%%
%% Create globally-available macros to be provided for style writers
%% These are redefined for each occurence of each division
\newcommand{\divisionnameptx}{\relax}%
%\newcommand{\titleptx}{\relax}%
\newcommand{\subtitleptx}{\relax}%
\newcommand{\shortitleptx}{\relax}%
\newcommand{\authorsptx}{\relax}%
\newcommand{\epigraphptx}{\relax}%
%% Create environments for possible occurences of each division
%% Environment for a PTX "acknowledgement" at the level of a LaTeX "chapter"
%\NewDocumentEnvironment{acknowledgement}{mmmmmm}
{%
%\renewcommand{\divisionnameptx}{Acknowledgements}%
\renewcommand{\titleptx}{#1}%
\renewcommand{\subtitleptx}{#2}%
\renewcommand{\shortitleptx}{#3}%
\renewcommand{\authorsptx}{#4}%
\renewcommand{\epigraphptx}{#5}%
\chapter*{#1}%
\addcontentsline{toc}{chapter}{#3}
\label{#6}%
}{}%
%% Environment for a PTX "preface" at the level of a LaTeX "chapter"
\NewDocumentEnvironment{preface}{mmmmmm}
{%
\renewcommand{\divisionnameptx}{Preface}%
\renewcommand{\titleptx}{#1}%
\renewcommand{\subtitleptx}{#2}%
\renewcommand{\shortitleptx}{#3}%
\renewcommand{\authorsptx}{#4}%
\renewcommand{\epigraphptx}{#5}%
\chapter*{#1}%
\addcontentsline{toc}{chapter}{#3}
\label{#6}%
}{}%
%% Environment for a PTX "chapter" at the level of a LaTeX "chapter"
\NewDocumentEnvironment{chapterptx}{mmmmmm}
{%
\renewcommand{\divisionnameptx}{Chapter}%
\renewcommand{\titleptx}{#1}%
\renewcommand{\subtitleptx}{#2}%
\renewcommand{\shortitleptx}{#3}%
\renewcommand{\authorsptx}{#4}%
\renewcommand{\epigraphptx}{#5}%
\chapter[#3]{#1}%
\label{#6}%
}{}%
%% Environment for a PTX "section" at the level of a LaTeX "section"
\NewDocumentEnvironment{sectionptx}{mmmmmm}
{%
\renewcommand{\divisionnameptx}{Section}%
\renewcommand{\titleptx}{#1}%
\renewcommand{\subtitleptx}{#2}%
\renewcommand{\shortitleptx}{#3}%
\renewcommand{\authorsptx}{#4}%
\renewcommand{\epigraphptx}{#5}%
\section[#3]{#1}%
\label{#6}%
}{}%
%% Environment for a PTX "subsection" at the level of a LaTeX "subsection"
\NewDocumentEnvironment{subsectionptx}{mmmmmm}
{%
\renewcommand{\divisionnameptx}{Subsection}%
\renewcommand{\titleptx}{#1}%
\renewcommand{\subtitleptx}{#2}%
\renewcommand{\shortitleptx}{#3}%
\renewcommand{\authorsptx}{#4}%
\renewcommand{\epigraphptx}{#5}%
\subsection[#3]{#1}%
\label{#6}%
}{}%
%% Environment for a PTX "exercises" at the level of a LaTeX "subsection"
\NewDocumentEnvironment{exercises-subsection}{mmmmmm}
{%
\renewcommand{\divisionnameptx}{Exercises}%
\renewcommand{\titleptx}{#1}%
\renewcommand{\subtitleptx}{#2}%
\renewcommand{\shortitleptx}{#3}%
\renewcommand{\authorsptx}{#4}%
\renewcommand{\epigraphptx}{#5}%
\subsection[#3]{#1}%
\label{#6}%
}{}%
%% Environment for a PTX "exercises" at the level of a LaTeX "subsection"
\NewDocumentEnvironment{exercises-subsection-numberless}{mmmmmm}
{%
\renewcommand{\divisionnameptx}{Exercises}%
\renewcommand{\titleptx}{#1}%
\renewcommand{\subtitleptx}{#2}%
\renewcommand{\shortitleptx}{#3}%
\renewcommand{\authorsptx}{#4}%
\renewcommand{\epigraphptx}{#5}%
\subsection*{#1}%
\addcontentsline{toc}{subsection}{#3}
\label{#6}%
}{}%
%% Environment for a PTX "index" at the level of a LaTeX "chapter"
\NewDocumentEnvironment{indexptx}{mmmmmm}
{%
\renewcommand{\divisionnameptx}{Index}%
\renewcommand{\titleptx}{#1}%
\renewcommand{\subtitleptx}{#2}%
\renewcommand{\shortitleptx}{#3}%
\renewcommand{\authorsptx}{#4}%
\renewcommand{\epigraphptx}{#5}%
\chapter*{#1}%
\addcontentsline{toc}{chapter}{#3}
\label{#6}%
}{}%
%%
%% Styles for the traditional LaTeX divisions
\titleformat{\chapter}[display]
{\normalfont\huge\bfseries}{\divisionnameptx\space\thechapter}{20pt}{\Huge#1}
[{\Large\authorsptx}]
\titleformat{name=\chapter,numberless}[display]
{\normalfont\huge\bfseries}{}{0pt}{#1}
[{\Large\authorsptx}]
\titlespacing*{\chapter}{0pt}{50pt}{40pt}
\titleformat{\chapter}[display]
{\raggedleft\normalfont\color{chaptercolor}\Large}{\MakeUppercase{\divisionnameptx}\space\rlap{\enskip\resizebox{!}{0.95cm}{\thechapter} \rule{15cm}{0.95cm}}}{10pt}{\normalfont\Huge\itshape#1}
[{\Large\authorsptx}]
\titleformat{name=\chapter,numberless}[display]
{\raggedleft\normalfont\color{chaptercolor}\Huge\itshape}{}{0pt}{#1}
[{\Large\authorsptx}]
\titlespacing*{\chapter}{0pt}{30pt}{20pt}
\titleformat{\section}[block]
{\normalfont\Large\bfseries}{\thesection\space\titleptx}{1em}{}
[{\large\authorsptx}]
\titleformat{name=\section,numberless}[block]
{\normalfont\Large\bfseries}{}{0pt}{#1}
[{\large\authorsptx}]
\titlespacing*{\section}{0pt}{3.5ex plus 1ex minus .2ex}{2.3ex plus .2ex}
\titleformat{\subsection}[block]
{\normalfont\large\bfseries}{\thesubsection\space\titleptx}{1em}{}
[{\normalsize\authorsptx}]
\titleformat{name=\subsection,numberless}[block]
{\normalfont\large\bfseries}{}{0pt}{#1}
[{\normalsize\authorsptx}]
\titlespacing*{\subsection}{0pt}{3.25ex plus 1ex minus .2ex}{1.5ex plus .2ex}
\titleformat{\subsubsection}[block]
{\normalfont\normalsize\bfseries}{\thesubsubsection\space\titleptx}{1em}{}
[{\small\authorsptx}]
\titleformat{name=\subsubsection,numberless}[block]
{\normalfont\normalsize\bfseries}{}{0pt}{#1}
[{\normalsize\authorsptx}]
\titlespacing*{\subsubsection}{0pt}{3.25ex plus 1ex minus .2ex}{1.5ex plus .2ex}
\titleformat{\subsection}[block]
{\normalfont\large\bfseries}{\thesubsection\space\titleptx}{1em}{}
[{\normalsize\authorsptx}]
\titleformat{name=\subsection,numberless}[block]
{\normalfont\large\bfseries}{}{0pt}{#1}
[{\normalsize\authorsptx}]
\titlespacing*{\subsection}{0pt}{3.25ex plus 1ex minus .2ex}{1.5ex plus .2ex}
\titleformat{\subsubsection}[block]
{\normalfont\normalsize\bfseries}{\thesubsubsection\space\titleptx}{1em}{}
[{\small\authorsptx}]
\titleformat{name=\subsubsection,numberless}[block]
{\normalfont\normalsize\bfseries}{}{0pt}{#1}
[{\normalsize\authorsptx}]
\titlespacing*{\subsubsection}{0pt}{3.25ex plus 1ex minus .2ex}{1.5ex plus .2ex}
%%
%% Semantic Macros
%% To preserve meaning in a LaTeX file
%% Only defined here if required in this document
%% Used for inline definitions of terms
\newcommand{\terminology}[1]{\textbf{#1}}
%% Used for fillin answer blank
%% Argument is length in em
%% Length may compress for output to fit in one line
\newcommand{\fillin}[1]{\leavevmode\leaders\vrule height -1.2pt depth 1.5pt \hskip #1em minus #1em \null}
%% Subdivision Numbering, Chapters, Sections, Subsections, etc
%% Subdivision numbers may be turned off at some level ("depth")
%% A section *always* has depth 1, contrary to us counting from the document root
%% The latex default is 3.  If a larger number is present here, then
%% removing this command may make some cross-references ambiguous
%% The precursor variable $numbering-maxlevel is checked for consistency in the common XSL file
\setcounter{secnumdepth}{3}
%% begin: General AMS environment setup
%% Environments built with amsthm package
\usepackage{amsthm}
%% Numbering for Theorems, Conjectures, Examples, Figures, etc
%% Controlled by  numbering.theorems.level  processing parameter
%% Numbering: all theorem-like numbered consecutively
%% i.e. Corollary 4.3 follows Theorem 4.2
%% Always need some theorem environment to set base numbering scheme
%% even if document has no theorems (but has other environments)
%% Create a never-used style first, always
%% simply to provide a global counter to use, namely "cthm"
\newtheorem{cthm}{BadTheoremStringName}[section]
%% AMS "proof" environment is not used, but we leave previously
%% implemented \qedhere in place, should the LaTeX be recycled
\renewcommand{\qedhere}{\relax}
%% end: General AMS environment setup
%%
%% tcolorbox, with styles, for DEFINITION-LIKE
%%
%% definition: fairly simple numbered block/structure
\tcbset{ definitionstyle/.style={bwminimalstyle, runintitlestyle, blockspacingstyle, after title={\space}, after upper={\hfill{}\(\lozenge\)}, } }
\newtcolorbox[use counter*=cthm]{definition}[2]{title={{Definition~\thecthm\notblank{#1}{\space\space#1}{}}}, phantomlabel={#2}, breakable, definitionstyle, }
%%
%% tcolorbox, with styles, for EXAMPLE-LIKE
%%
%% example: fairly simple numbered block/structure
\tcbset{ examplestyle/.style={bwminimalstyle, runintitlestyle, blockspacingstyle, after title={\space}, after upper={\hfill{}\(\square\)}, } }
\newtcolorbox[use counter*=cthm]{example}[2]{title={{Example~\thecthm\notblank{#1}{\space\space#1}{}}}, phantomlabel={#2}, breakable, examplestyle, }
%%
%% tcolorbox, with styles, for PROJECT-LIKE
%%
%% exploration: fairly simple numbered block/structure
\tcbset{ explorationstyle/.style={enhanced,frame hidden,interior hidden, sharp corners,
boxrule=0pt,borderline west={3pt}{0pt}{ActiveBlue}, 
runintitlestyle, blockspacingstyle, after title={.\space}, 
after upper={\hfill{}\(\square\)}, colback=white,
coltitle=black,} }
\newtcolorbox[use counter*=cpjt]{exploration}[2]{title={{Preview Activity~\thecpjt\notblank{#1}{\space\space#1}{}}}, phantomlabel={#2}, breakable, explorationstyle, }
%% activity: fairly simple numbered block/structure
\tcbset{ activitystyle/.style={enhanced,frame hidden,interior hidden, sharp corners,
boxrule=0pt,borderline west={3pt}{0pt}{ActiveBlue}, 
runintitlestyle, blockspacingstyle, after title={.\space}, 
after upper={\hfill{}\(\square\)}, colback=white,
coltitle=black,} }
\newtcolorbox[use counter*=cpjt]{activity}[2]{title={{Activity~\thecpjt\notblank{#1}{\space\space#1}{}}}, phantomlabel={#2}, breakable, activitystyle, }
%%
%% xparse environments for introductions and conclusions of divisions
%%
%% introduction: in a structured division
\NewDocumentEnvironment{introduction}{m}
{\notblank{#1}{\noindent\textbf{#1}\space}{}}{\par\medskip}
%%
%% tcolorbox, with styles, for miscellaneous environments
%%
%% objectives: early in a subdivision, introduction/list/conclusion
\tcbset{ objectivesstyle/.style={enhanced,frame hidden,interior hidden,sharp corners,
blockspacingstyle,boxrule=0pt,left=0pt,right=0pt,
fonttitle=\large\bfseries,
borderline north={0.1ex}{0pt}{black},
toptitle=0.5ex,top=2ex, bottom=0.5ex, 
borderline south={0.1ex}{0pt}{black},coltitle=black,} }
\newtcolorbox{objectives}[2]{title={#1}, phantomlabel={#2}, breakable, objectivesstyle}
%% assemblage: fairly simple un-numbered block/structure
\tcbset{ assemblagestyle/.style={breakable, skin=enhanced, arc=2ex, 
colback=ActiveBlue!5,colframe=ActiveBlue!75!black, 
colbacktitle=ActiveBlue!20, coltitle=black, 
boxed title style={sharp corners, frame hidden}, 
fonttitle=\bfseries, attach boxed title to top 
left={xshift=4mm,yshift=-3mm}, top=3mm,
} }
\newtcolorbox{assemblage}[2]{title={\notblank{#1}{#1}{}}, phantomlabel={#2}, breakable, assemblagestyle}
%% back colophon, at the very end, typically on its own page
\tcbset{ backcolophonstyle/.style={bwminimalstyle, blockspacingstyle, before skip=5ex, left skip=0.15\textwidth, right skip=0.15\textwidth, fonttitle=\large\bfseries, center title, halign=center, bottomtitle=2ex} }
\newtcolorbox{backcolophon}[1]{title={Colophon}, phantom={\hypertarget{#1}{}}, breakable, backcolophonstyle}
%% Numbering for Projects (independent of others)
%% Controlled by  numbering.projects.level  processing parameter
%% Always need a project environment to set base numbering scheme
%% even if document has no projectss (but has other blocks)
%% So "cpjt" environment produces "cpjt" counter
\newtheorem{cpjt}{BadProjectNameString}[section]
%% Divisional exercises (and worksheet) as LaTeX environments
%% Third argument is option for extra workspace in worksheets
%% Hanging indent occupies a 5ex width slot prior to left margin
%% Experimentally this seems just barely sufficient for a bold "888."
%% Division exercises, not in exercise group
\tcbset{ divisionexercisestyle/.style={bwminimalstyle, runintitlestyle, exercisespacingstyle, left=5ex, breakable } }
\newtcolorbox{divisionexercise}[4]{divisionexercisestyle, before title={\hspace{-5ex}\makebox[5ex][l]{#1.}}, title={\notblank{#2}{#2\space}{}}, phantom={\hypertarget{#4}{}}, after={\notblank{#3}{\newline\rule{\workspacestrutwidth}{#3\textheight}\newline}{}}}
%% Localize LaTeX supplied names (possibly none)
\renewcommand*{\chaptername}{Chapter}
%% Equation Numbering
%% Controlled by  numbering.equations.level  processing parameter
\numberwithin{equation}{section}
%% For improved tables
\usepackage{array}
%% Some extra height on each row is desirable, especially with horizontal rules
%% Increment determined experimentally
\setlength{\extrarowheight}{0.2ex}
%% Define variable thickness horizontal rules, full and partial
%% Thicknesses are 0.03, 0.05, 0.08 in the  booktabs  package
\makeatletter
\newcommand{\hrulethin}  {\noalign{\hrule height 0.04em}}
\newcommand{\hrulemedium}{\noalign{\hrule height 0.07em}}
\newcommand{\hrulethick} {\noalign{\hrule height 0.11em}}
%% We preserve a copy of the \setlength package before other
%% packages (extpfeil) get a chance to load packages that redefine it
\let\oldsetlength\setlength
\newlength{\Oldarrayrulewidth}
\newcommand{\crulethin}[1]%
{\noalign{\global\oldsetlength{\Oldarrayrulewidth}{\arrayrulewidth}}%
\noalign{\global\oldsetlength{\arrayrulewidth}{0.04em}}\cline{#1}%
\noalign{\global\oldsetlength{\arrayrulewidth}{\Oldarrayrulewidth}}}%
\newcommand{\crulemedium}[1]%
{\noalign{\global\oldsetlength{\Oldarrayrulewidth}{\arrayrulewidth}}%
\noalign{\global\oldsetlength{\arrayrulewidth}{0.07em}}\cline{#1}%
\noalign{\global\oldsetlength{\arrayrulewidth}{\Oldarrayrulewidth}}}
\newcommand{\crulethick}[1]%
{\noalign{\global\oldsetlength{\Oldarrayrulewidth}{\arrayrulewidth}}%
\noalign{\global\oldsetlength{\arrayrulewidth}{0.11em}}\cline{#1}%
\noalign{\global\oldsetlength{\arrayrulewidth}{\Oldarrayrulewidth}}}
%% Single letter column specifiers defined via array package
\newcolumntype{A}{!{\vrule width 0.04em}}
\newcolumntype{B}{!{\vrule width 0.07em}}
\newcolumntype{C}{!{\vrule width 0.11em}}
\makeatother
%% Figures, Tables, Listings, Named Lists, Floats
%% The [H]ere option of the float package fixes floats in-place,
%% in deference to web usage, where floats are totally irrelevant
%% You can remove some of this setup, to restore standard LaTeX behavior
%% HOWEVER, numbering of figures/tables AND theorems/examples/remarks, etc
%% may de-synchronize with the numbering in the HTML version
%% You can remove the "placement={H}" option to allow flotation and
%% preserve numbering, BUT the numbering may then appear "out-of-order"
%% Floating environments: http://tex.stackexchange.com/questions/95631/
\usepackage{float}
\usepackage{newfloat}
\usepackage{caption}%% Adjust stock figure environment so that it no longer floats
\SetupFloatingEnvironment{figure}{fileext=lof,placement={H},within=section,name=Figure}
\captionsetup[figure]{labelfont=bf}
%% http://tex.stackexchange.com/questions/16195
\makeatletter
\let\c@figure\c@cthm
\makeatother
%% Adjust stock table environment so that it no longer floats
\SetupFloatingEnvironment{table}{fileext=lot,placement={H},within=section,name=Table}
\captionsetup[table]{labelfont=bf}
%% http://tex.stackexchange.com/questions/16195
\makeatletter
\let\c@table\c@cthm
\makeatother
%% Footnote Numbering
%% We reset the footnote counter, as given by numbering.footnotes.level
\makeatletter\@addtoreset{footnote}{section}\makeatother
%% QR Code Support
%% Videos and other interactives
\usepackage{qrcode}
\newlength{\qrsize}
\newlength{\previewwidth}
%% tcolorbox styles for interactive previews
%% changing size= and/or colback can aid in debugging
\tcbset{ previewstyle/.style={bwminimalstyle, halign=center} }
\tcbset{ qrstyle/.style={bwminimalstyle, hbox} }
\tcbset{ captionstyle/.style={bwminimalstyle, left=1em, width=\linewidth} }
%% Generic red play button (from SVG)
%% tikz package should be loaded by now
\definecolor{playred}{RGB}{230,33,23}
\newcommand{\genericpreview}{
        \begin{tikzpicture}[y=0.80pt, x=0.80pt, yscale=-1.000000, xscale=1.000000, inner sep=0pt, outer sep=0pt]
        \path[fill=playred] (94.9800,28.8400) .. controls (94.9800,28.8400) and
        (94.0400,22.2400) .. (91.1700,19.3400) .. controls (87.5300,15.5300) and
        (83.4500,15.5100) .. (81.5800,15.2900) .. controls (68.1800,14.3200) and
        (48.0600,14.4400) .. (48.0600,14.4400) .. controls (48.0600,14.4400) and
        (27.9400,14.3200) .. (14.5400,15.2900) .. controls (12.6700,15.5100) and
        (8.5900,15.5300) .. (4.9500,19.3400) .. controls (2.0800,22.2400) and
        (1.1400,28.8400) .. (1.1400,28.8400) .. controls (1.1400,28.8400) and
        (0.1800,36.5800) .. (0.0000,44.3300) -- (0.0000,51.5900) .. controls
        (0.1800,59.3400) and (1.1400,67.0800) .. (1.1400,67.0800) .. controls
        (1.1400,67.0800) and (2.0700,73.6800) .. (4.9500,76.5800) .. controls
        (8.5900,80.3900) and (13.3800,80.2700) .. (15.5100,80.6700) .. controls
        (23.0400,81.3900) and (47.2100,81.5600) .. (48.0500,81.5700) .. controls
        (48.0600,81.5700) and (68.1900,81.6000) .. (81.5900,80.6300) .. controls
        (83.4600,80.4100) and (87.5400,80.3900) .. (91.1800,76.5800) .. controls
        (94.0500,73.6800) and (94.9900,67.0800) .. (94.9900,67.0800) .. controls
        (94.9900,67.0800) and (95.9500,59.3300) .. (96.0100,51.5900) --
        (96.0100,44.3300) .. controls (95.9400,36.5800) and (94.9800,28.8400) ..
        (94.9800,28.8400) -- cycle(38.2800,61.4100) -- (38.2800,34.4100) --
        (64.0200,47.9100) -- (38.2800,61.4100) -- cycle;
        \end{tikzpicture}
        }
%% Multiple column, column-major lists
\usepackage{multicol}
%% More flexible list management, esp. for references
%% But also for specifying labels (i.e. custom order) on nested lists
\usepackage{enumitem}
%% Support for index creation
%% imakeidx package does not require extra pass (as with makeidx)
%% Title of the "Index" section set via a keyword
%% Language support for the "see" and "see also" phrases
\usepackage{imakeidx}
\makeindex[title=Index, intoc=true]
\renewcommand{\seename}{see}
\renewcommand{\alsoname}{see also}
%% hyperref driver does not need to be specified, it will be detected
\usepackage{hyperref}
%% configure hyperref's  \url  to match listings' inline verbatim
\renewcommand\UrlFont{\small\ttfamily}
%% Hyperlinking active in PDFs, all links solid and blue
\hypersetup{colorlinks=true,linkcolor=blue,citecolor=blue,filecolor=blue,urlcolor=blue}
\hypersetup{pdftitle={Active Preparation for Calculus}}
%% If you manually remove hyperref, leave in this next command
\providecommand\phantomsection{}
%% If tikz has been loaded, replace ampersand with \amp macro
%% tcolorbox styles for sidebyside layout
\tcbset{ sbsstyle/.style={raster equal height=rows,raster force size=false} }
\tcbset{ sbsheadingstyle/.style={bwminimalstyle, halign=center, fontupper=\bfseries} }
\tcbset{ sbspanelstyle/.style={bwminimalstyle} }
\tcbset{ sbscaptionstyle/.style={bwminimalstyle, halign=center} }
%% Enviroments for side-by-side and components
%% Necessary to use \NewTColorBox for boxes of the panels
%% "newfloat" environment to squash page-breaks within a single sidebyside
%% \leavevmode necessary when a side-by-side comes first, right after a heading
%% "xparse" environment for entire sidebyside
\NewDocumentEnvironment{sidebyside}{mmmm}
  {\begin{tcbraster}
    [sbsstyle,raster columns=#1,
    raster left skip=#2\linewidth,raster right skip=#3\linewidth,raster column skip=#4\linewidth]}
  {\end{tcbraster}}
%% "tcolorbox" environments for three components of a panel
\NewTColorBox{sbsheading}{m}{sbsheadingstyle,width=#1\linewidth}
\NewTColorBox{sbspanel}{mO{top}}{sbspanelstyle,width=#1\linewidth,valign=#2}
\NewTColorBox{sbscaption}{m}{sbscaptionstyle,width=#1\linewidth}
%% extpfeil package for certain extensible arrows,
%% as also provided by MathJax extension of the same name
%% NB: this package loads mtools, which loads calc, which redefines
%%     \setlength, so it can be removed if it seems to be in the 
%%     way and your math does not use:
%%     
%%     \xtwoheadrightarrow, \xtwoheadleftarrow, \xmapsto, \xlongequal, \xtofrom
%%     
%%     we have had to be extra careful with variable thickness
%%     lines in tables, and so also load this package late
\usepackage{extpfeil}
%% Custom Preamble Entries, late (use latex.preamble.late)
%% Used to get WeBWorK logo into margin next to WW exercises
\usepackage{marginnote}
%%%%%%%%%%%%%%%%%%%%%%%%%%%%%%%%%%%%%%%%
% CC icon at bottom of first page of each chapter 
%%%%%%%%%%%%%%%%%%%%%%%%%%%%%%%%%%%%%%%%
\newpagestyle{chapopen}{
\sethead[][][] % even
{}{}{} % odd
\setfoot[\includegraphics[height=1pc]{images/CC-BY-SA-license.pdf}][][]
{}{}{\includegraphics[height=1pc]{images/CC-BY-SA-license.pdf}}}
\assignpagestyle{\chapter}{chapopen}
%%%%%%%%%%%%%%%%%%%%%%%%%%%%%%%%%%%%%%%%
% Modified from Mitch Keller's chapter handling 
%%%%%%%%%%%%%%%%%%%%%%%%%%%%%%%%%%%%%%%%
%%% This is from common
\definecolor{ActiveBlue}{cmyk}{1, 0.5, 0, 0.35}
\colorlet{chaptercolor}{ActiveBlue}
%%%%%%%%%%%%%%%%%%%%%%%%%%%%%%%%%%%%%%%%
% Basic paragraph parameters 
%%%%%%%%%%%%%%%%%%%%%%%%%%%%%%%%%%%%%%%%
\setlength{\parindent}{0mm}
\setlength{\parskip}{0.5pc}
%%%%%%%%%%%%%%%%%%%%%%%%%%%%%%%%%%%%%%%%
% In print, trying to reduce color use 
%%%%%%%%%%%%%%%%%%%%%%%%%%%%%%%%%%%%%%%%
\hypersetup{colorlinks=true,linkcolor=black,citecolor=black,filecolor=black,urlcolor=black}
%%%%%%%%%%%%%%%%%%%%%%%%%%%%%%%%%%%%%%%%
% Start sections on new page, just not the first one 
%%%%%%%%%%%%%%%%%%%%%%%%%%%%%%%%%%%%%%%%
\let\oldsection\section 
\renewcommand\section{\znewpage\oldsection}

\let\oldchapter\chapter 
\renewcommand\chapter{\clearpage\gdef\znewpage{\global\let\znewpage\clearpage}\oldchapter}

\global\let\znewpage\clearpage 

%% Begin: Author-provided packages
%% (From  docinfo/latex-preamble/package  elements)
%% End: Author-provided packages
%% Begin: Author-provided macros
%% (From  docinfo/macros  element)
%% Plus three from MBX for XML characters

\newcommand{\lt}{<}
\newcommand{\gt}{>}
\newcommand{\amp}{&}
%% End: Author-provided macros